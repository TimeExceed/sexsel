\documentclass[UTF8,lualatex]{ctexbeamer}
\usepackage{graphicx}
\usepackage{tabu}
\usepackage{amsmath}

\usetheme{Frankfurt}
\setbeamertemplate{note page}[plain]
% $BEAMER_NOTES

\title{\kaishu 性选择与演化}
    \author{陶大}
    \institute{纳拓软件}
    \date{\tiny
        Created in Jan.\ 2011\\
        Updated in May\ 2021}

\begin{document}
\songti
    
\begin{frame}
\titlepage
\end{frame}

\section{什么是演化论}

\begin{frame}
    \frametitle{什么是演化论}
    \begin{block}{什么是演化论?}
        一个生活在35亿多年前的原始物种——可能是一个能够自我复制的分子——逐步演化出了地球上的所有物种,
        其规模随着时间而不断扩大,发散出许许多多新的不同物种,其中所发生的大多数(不是全部)演化改变的机制是自然选择。
        \begin{itemize}
            \item<2-> 演化
            \item<2-> 渐进
            \item<2-> 物种形成
            \item<2-> 共同祖先
            \item<2-> 自然选择
            \item<2-> 其他演化机制
                \begin{itemize}
                    \item<3> \textcolor{red}{性选择}
                \end{itemize}
        \end{itemize}
    \end{block}
\end{frame}

\note{
    \begin{itemize}
        \item 演化,不是“进化”。演化没有方向。
        \item 渐进,每次都只动局部。不搞全局推翻重来。
        \item 自然选择只是其中之一,性选择和中性漂变也起了作用。
    \end{itemize}
}

\section{性选择}
\subsection{性别二态性}

\begin{frame}
    \frametitle{孔雀难题}
    \begin{columns}
        \begin{column}{.47\textwidth}
            \includegraphics[width=\textwidth]{peafowl.jpg}
            \\
            求偶中的雄孔雀
            \\ \small
            (src: \url{https://e-info.org.tw/node/84415})
        \end{column}
        \begin{column}{.47\textwidth}
            \begin{itemize}
                \item 雌雄个体在外观上的差异
                \item 雄孔雀的尾巴是个累赘
                    \begin{itemize}
                        \item<2> Then, WHY?
                    \end{itemize}
            \end{itemize}
        \end{column}
    \end{columns}
\end{frame}

\note{
    \begin{itemize}
        \item 雨季里就是条拖把
        \item 不利其飞行
        \item 鲜亮的颜色吸引捕食者
        \item 占用大量的能量
    \end{itemize}
}

\begin{frame}
    \frametitle{爱尔兰马鹿}
    \begin{columns}
        \begin{column}{0.48\textwidth}
            \includegraphics[width=\textwidth,height=5cm]{male-irish-dear.jpg}
            \\ \small
            Male Irish Dear
        \end{column}
        \begin{column}{0.48\textwidth}
            \includegraphics[width=\textwidth,height=5cm]{female-irish-dear.jpg}
            \\ \small
            Female Irish Dear
        \end{column}
    \end{columns}
    
    \begin{center}
        \small
        (src: \url{https://en.wikipedia.org/wiki/Irish_elk})    
    \end{center}
\end{frame}

\note{
    \begin{itemize}
        \item 爱尔兰马鹿灭绝于约一万年前。
            既不是爱尔兰特有(曾经遍布欧亚大陆),也不是一种马鹿。
        \item 只有雄鹿长有巨角。
        \item 巨角跨度近4米,重量可达40公斤。作为对照,其头颅仅有2公斤。
        \item 更过分的是,每年都要脱落再重新生长出来。
        \item Then, WHY?
    \end{itemize}
}

\begin{frame}
    \frametitle{“不适应性”}
    \begin{columns}
        \begin{column}{.48\textwidth}
            \includegraphics[width=\textwidth]{male-euplectes-ardens.jpeg}
        \end{column}
        \begin{column}{.48\textwidth}
            \includegraphics[width=\textwidth]{shorten_euplectes-ardens.jpeg}
        \end{column}
    \end{columns}

    \begin{center}
        红领寡妇鸟
    \end{center}
\end{frame}

\note{
    \begin{itemize}
        \item 剪短红领寡妇鸟的尾羽,反而使得雄鸟的身体健康状况得到普遍的改善。
    \end{itemize}
}

\begin{frame}
    \frametitle{“不适应性”}
    \begin{columns}[t]
        \begin{column}{.48\textwidth}
            \includegraphics[width=\textwidth]{crotaphytus-collaris.jpeg}
            \\
            北京动物园的环颈蜥
        \end{column}
        \begin{column}{.48\textwidth}
            \begin{tabular}{*{5}{c}}
                \hline
                    & S1  & S2  & S3    & Total \\
                \hline
                M1  & 14  & 3   & 1     & 18 \\
                M2  & 4   & 1   & 0     & 5 \\
                M3  & 8   & 2   & 2     & 12 
                \onslide<2->{
                \\ \hline
                F   & 0   & 0   & 0     & 0
                \\ \hline
                }
            \end{tabular}
        \end{column}
    \end{columns}
\end{frame}

\note{
    \begin{itemize}
        \item 图中是一对环颈蜥。雄蜥体色鲜亮而雌蜥黯淡。
        \item 环颈蜥有三种生境,雄蜥也有三种体色。
            生物学家在每种生境中都挑选了各种体色的雄蜥各若干,
            然后跟踪观察到实验期结束被捕食的各有多少。
            他们画了一张表格,$S_i$代表生境,$M_j$代表某种体色的雄蜥,格子代表在$M_j$在$S_i$生境中被捕食的数量。
        \item 那么雌蜥如何呢?
            Zero!
        \item Then, 雄蜥干嘛把自己搞那么扎眼呢?
    \end{itemize}
}

\begin{frame}
    \frametitle{性别二态性}
    \begin{block}{性别二态性(Sexual Dimorphism)}
        \begin{itemize}
            \item 雌雄个体的形态、行为差异巨大
            \item 雄性之形态、行为有害其自身
                \begin{itemize}
                    \item 消耗能量、不便生活
                    \item 更易被捕食
                    \item 更倾向于冒险
                    \item ……
                \end{itemize}
        \end{itemize}
    \end{block}
\end{frame}

\subsection{性选择}

\begin{frame}
    \frametitle{性选择}
    \begin{block}{性选择}
        一个性状,如果
        \begin{itemize}
            \item 使其所有者拥有后代的机会,
            \item 超过有害个体生存而损失的拥有后代的机会,
        \end{itemize}
        那么这个性状背后的基因就会传播开来。
    \end{block}
    \begin{block}<2>{性选择的基本形式}
        \begin{itemize}
            \item 争斗法则:雄性之间发生“面对面”的冲突
            \item 配偶选择:雌性偏爱某些雄性的特征或行为
        \end{itemize}
    \end{block}
\end{frame}

\subsubsection{争斗法则}

\begin{frame}
    \frametitle{争斗法则}
    \begin{columns}[t]
        \begin{column}{.48\textwidth}
            \includegraphics[width=\textwidth]{sexual-dimorphism_northern-elephant-seal.jpg}
            \\
            \tiny
            (src: \url{https://upload.wikimedia.org/wikipedia/commons/d/d8/See_elefanten_edit.jpg})
        \end{column}
        \begin{column}{.48\textwidth}
            \includegraphics[width=\textwidth]{male-fighting_northern-elephant-seal.jpg}
            \\
            \tiny
            (src: \url{https://sa.ylib.com/MagArticle.aspx?id=4815})
        \end{column}
    \end{columns}

    \begin{center}
        \small 北象海豹
    \end{center}
\end{frame}

\note{
    \begin{itemize}
        \item 一夫多妻制
        \item 雄性为获得交配机会激烈打斗
        \item 结果:雄性的体型极为巨大
            \begin{itemize}
                \item 体长6米
                \item 体重3吨
            \end{itemize}
    \end{itemize}
}

\begin{frame}
    \frametitle{争斗法则}
    \begin{columns}[t]
        \begin{column}{.48\textwidth}
            \includegraphics[width=\textwidth]{red-winged-blackbird.jpg}
        \end{column}
        \begin{column}{.48\textwidth}
            \includegraphics[width=\textwidth]{red-winged-blackbird_flock.jpg}
        \end{column}
    \end{columns}

    \begin{center}
        \small 美洲红翼鸫
        \\
        \tiny
        (src: \url{https://en.wikipedia.org/wiki/Red-winged_blackbird})
    \end{center}
\end{frame}

\note{
    \begin{itemize}
        \item 一夫多妻制
        \item 然而情敌于其生存有利
        \item 所以,“要文斗不要武斗”。
            \begin{itemize}
                \item 肩羽:失去艳丽肩章的雄性有70\%失去领地,对照组仅有10\%。
                \item 歌声:失去发声能力的雄性也同样失去自己的领地。
            \end{itemize}
    \end{itemize}
}

\begin{frame}
    \frametitle{争斗法则}
    \begin{columns}[t]
        \begin{column}{.48\textwidth}
            \begin{center}
                \includegraphics[width=4cm,height=3cm]{millipede.jpg}
            \end{center}
        \end{column}
        \begin{column}{.48\textwidth}
            \begin{center}
                \includegraphics[width=4cm,height=3cm]{damselfly.jpg}
            \end{center}
        \end{column}
    \end{columns}
    \begin{columns}[t]
        \begin{column}{.48\textwidth}
            \begin{center}
                \includegraphics[width=4cm,height=3cm]{fruit-fly.jpg}
            \end{center}
        \end{column}
        \begin{column}{.48\textwidth}
            \begin{center}
                \includegraphics[width=4cm,height=3cm]{snake.jpg}
            \end{center}
        \end{column}
    \end{columns}
\end{frame}

\note{
    \begin{itemize}
        \item 马陆:雄性交配后骑在雌性身上不下来
        \item 豆娘:阴茎上的倒刺可以将其他雄性留下的精液刮出来
        \item 果蝇:某些种类的精液里含有抑情剂和使其他果蝇精子失活的杀精剂
        \item 蛇:某些种类的精液可以暂时堵塞雌性的阴道
    \end{itemize}
}

\subsubsection{配偶选择}

\begin{frame}
    \frametitle{配偶选择}
    \begin{columns}
        \begin{column}{.48\textwidth}
            \includegraphics[width=\textwidth]{peafowl_plumage.jpg}
        \end{column}
        \begin{column}{.48\textwidth}
            \includegraphics[width=\textwidth]{long-tailed-widowbird.jpg}
        \end{column}
    \end{columns}
\end{frame}

\note{
    \begin{itemize}
        \item 孔雀
            \begin{itemize}
                \item 最精美的雄孔雀有160个眼点,参与了36\%的交配。
                \item 人工减去20个眼点之后,比对照组平均减少2.5个配偶。
            \end{itemize}
        \item 长尾寡妇鸟
            \begin{itemize}
                \item 正常雄鸟尾长达半米。雌鸟仅为8厘米。
                \item 剪短尾羽的雄鸟几乎没有配偶。
                \item 接长尾羽的雄鸟吸引到的雌鸟几乎是对照组的2倍。
            \end{itemize}
    \end{itemize}
}

\section{性选择与演化}

\begin{frame}
    \frametitle{性选择与演化}
    \begin{block}{两个疑问}
        \begin{itemize}
            \item 为什么雄性争斗、雌性选择?
            \item 雌性选择什么?
        \end{itemize}
    \end{block}
\end{frame}

\subsection{雄性争斗、雌性选择}

\begin{frame}
    \frametitle{一些事实}
    \begin{columns}
        \begin{column}{.48\textwidth}
            \begin{block}{谁是生育最多的女性?}
                \onslide<2->{
                    某俄国妇女。
                    1725-1745二十年间怀孕27次,总计69个子女。
                    \begin{itemize}
                        \item 16次双胞胎
                        \item 7次三胞胎
                        \item 4次四胞胎
                    \end{itemize}
                }
            \end{block}
        \end{column}
        \begin{column}{.48\textwidth}
            \onslide<3>{
                \includegraphics[width=\textwidth]{mulai-ismail.jpg}
                \\
                \small
                Mulai Ismail, 18世纪的摩洛哥苏丹。
                至少342个女儿与525个儿子的父亲。
            }
        \end{column}
    \end{columns}
\end{frame}

\begin{frame}
    \frametitle{交配的成本与策略}
    \begin{block}{交配的机会成本}
        \begin{center}
            \begin{tabular}{*{3}{c}}
                \\ \hline
                        & 雌性  & 雄性 \\
                \hline
                配子    & 宝贵  & 廉价 \\
                怀孕    & 费时费力 & — \\
                抚养子女& 多数物种 & 少数物种 \\
                \hline
            \end{tabular}
        \end{center}
    \end{block}
    \begin{block}<2>{交配的策略}
        \begin{itemize}
            \item 雌性挑剔
            \item 雄性不加检点
        \end{itemize}
    \end{block}
\end{frame}

\note{
    \begin{itemize}
        \item 恐龙蛋,那么大个,也是一个卵子。
            卵子不仅提供遗传物质,也提供发育所需的养分。
    \end{itemize}
}

\begin{frame}
    \frametitle{雄性不加检点}
    \begin{columns}
        \begin{column}{.48\textwidth}
            \begin{center}
                \includegraphics[width=\textwidth]{centrocercus.jpg}
                \\
                \small
                艾松鸡
            \end{center}
        \end{column}
        \begin{column}{.48\textwidth}
            \begin{center}
                \includegraphics[width=\textwidth]{mormodes_and_bees.jpg}
                \\
                \small
                旋柱兰与蜜蜂
            \end{center}
        \end{column}
    \end{columns}
\end{frame}

\note{
    \begin{itemize}
        \item 艾松鸡以其复杂的求偶仪式闻名。
            较少人知道,雄鸡精虫上脑的时候甚至可以和牛粪“交配”。
        \item 雄蜂只要长得像雌蜂就上。
            旋柱兰利用这点,把自己的花柱长得像雌蜂,以此吸引雄蜂来帮助自己传播花粉。
    \end{itemize}
}

\begin{frame}
    \frametitle{性选择的涵义}
    \begin{block}{~}
        \begin{enumerate}
            \item 雌性选择
            \item 雄性竞争
            \item 性别二态性的物种中,几乎所有雌性都有配偶,大多数雄性没有配偶
        \end{enumerate}
    \end{block}
\end{frame}

\note{
    \begin{itemize}
        \item 雄性麋鹿的子代数量差异是雌性的3倍
        \item 不到10\%的雄性象海豹留下子代,而雌性超过一半
        \item 而大多数物种的出生性别比接近1:1,这意味着,对种群延续来讲,大多数雄性是无价值的。
            所以,雄性更好斗、更爱冒险(作死)。
    \end{itemize}
}

\begin{frame}
    \frametitle{一个随堂测验}
    \small
    \begin{columns}
        \begin{column}{.48\textwidth}
            \begin{center}
                \includegraphics[height=3cm]{mandarin-duck.jpg}
                \\
                鸳鸯
            \end{center}
        \end{column}
        \begin{column}{.48\textwidth}
            \begin{center}
                \includegraphics[height=3cm]{lovebird.jpg}
                \\
                爱情鸟
            \end{center}
        \end{column}
    \end{columns}
    \begin{columns}
        \begin{column}{.48\textwidth}
            \begin{center}
                \includegraphics[height=3cm]{anser.jpg}
                \\
                大雁
            \end{center}
        \end{column}
        \begin{column}{.48\textwidth}
            \begin{center}
                \includegraphics[height=3cm]{otididae.jpg}
                \\
                大鸨
            \end{center}
        \end{column}
    \end{columns}
\end{frame}

\note{
    \begin{itemize}
        \item 问题:哪个物种实行真一夫一妻制?
        \item 鸳鸯
            \begin{itemize}
                \item 中国传统的爱情鸟,“鸳鸯双宿又双飞”。
                \item 然而生物学家的观察发现,雌鸟只要一有机会就会“出轨”(当然雄鸟也不是好“鸟”)。
                    亲子鉴定表明,一窝雏鸟大约只有三分之一是雄鸟亲生的。
                \item “双宿又双飞”其实是盯梢。
            \end{itemize}
        \item 爱情鸟
            \begin{itemize}
                \item 西方的爱情鸟,literally。
            \end{itemize}
        \item 大雁
            \begin{itemize}
                \item 元好问的“问世间情为何物,直教生死相许”说的就是大雁。
                \item 中国传统结婚要送一对大雁。捉野生雁比较麻烦,所以后来改送鹅了。
                \item 雁确实真一夫一妻制。
                    但《雁邱词》里描述的自杀殉情场面是不可能的。
                    丧偶一方因为有育雏经验,其实更受欢迎。
            \end{itemize}
        \item 大鸨
            \begin{itemize}
                \item 《本草纲目》说大鸨“纯雌无雄,与他鸟合”。
                    故以鸨指妓。
                \item 其实只是大鸨的性别二态性太厉害,古人不识雄鸨。
            \end{itemize}
    \end{itemize}
}

\begin{frame}
    \frametitle{极限检验}
    \begin{columns}
        \begin{column}{.48\textwidth}
            \begin{center}
                \includegraphics[width=\textwidth,angle=90]{seahorse.jpg}
                \\
                \small
                怀孕中的雄海马
            \end{center}
        \end{column}
        \begin{column}{.48\textwidth}
            \begin{itemize}
                \item 生活在海草丛中。有些品种体型很小。
                \item 生产卵子耗能极大。雌鱼无力产生大量卵子。
                \item 必须走精养路线。
                \item 然而雌性无力孵育受精卵。所以雄性“怀孕”。
                \item 某些品种雄性的投入甚至大过雌性。
                \item<2> 雌性竞争:雌性更明艳、具有饰物。
            \end{itemize}
        \end{column}
    \end{columns}
\end{frame}

\subsection{雌性选择什么?}

\begin{frame}
    \frametitle{雌性选择什么?}
    \begin{columns}
        \begin{column}{.48\textwidth}
            \includegraphics[width=\textwidth]{peafowl.jpg}
        \end{column}
    \end{columns}
\end{frame}

\note{
    有两种理论
    \begin{enumerate}
        \item 达尔文理论:雌性不知何故偏爱某一种性状。
            雄性投其所好不断强化这种形状。
        \item 诚实信号理论:雄性通过某种不利自身生存的性状表明自身生存能力的强大。
            “瞧!老子这么自残还没挂。牛逼吧?”
    \end{enumerate}
    这两种理论并非互相排斥。

    比如男性的身高,故事可能是这样的:
    \begin{enumerate}
        \item 一开始,大家都穷。身高和营养高度相关。
            所以,男性身高是对家庭资源的诚实信号而被女性所偏好。
        \item 然后,工业革命了、绿色革命了,大家都能吃饱吃好。
            身高和家庭资源的相关性消失。
        \item 但是女性的偏好已成。所以仍然推动男性一代一代拔个子。
    \end{enumerate}
}

\section{应用于人类}

\begin{frame}
    \frametitle{应用于人类}
    \begin{block}{必须特别谨慎}
        \begin{itemize}
            \item 很难做实验
            \item 有些性别差异有其他解释:身高、力量
            \item 有时繁殖需要不那么强大:丁克家庭
        \end{itemize}
    \end{block}
\end{frame}

\begin{frame}
    \frametitle{应用于人类}
    \begin{center}
        \Huge
        未完待续!        
    \end{center}
\end{frame}

\end{document}
